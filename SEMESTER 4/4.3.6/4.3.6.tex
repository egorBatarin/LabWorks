%!TEX TS-program = xelatex
 
% Этот шаблон документа разработан в 2014 году
% Данилом Фёдоровых (danil@fedorovykh.ru) 
% для использования в курсе 
% <<Документы и презентации в \LaTeX>>, записанном НИУ ВШЭ
% для Coursera.org: http://coursera.org/course/latex .
% Исходная версия шаблона --- 
% https://www.writelatex.com/coursera/latex/5.2.2
 
\documentclass[a4paper,12pt]{article}
 
%%% Работа с русским языком
\usepackage[english,russian]{babel}   %% загружает пакет многоязыковой вёрстки
\usepackage{fontspec}      %% подготавливает загрузку шрифтов Open Type, True Type и др.
\defaultfontfeatures{Ligatures={TeX},Renderer=Basic}  %% свойства шрифтов по умолчанию
\setmainfont[Ligatures={TeX,Historic}]{Times New Roman} %% задаёт основной шрифт документа
\setsansfont{Comic Sans MS}                    %% задаёт шрифт без засечек
\setmonofont{Courier New}
\usepackage{indentfirst}
\frenchspacing
 
%%% Дополнительная работа с математикой
\usepackage{amsmath,amsfonts,amssymb,amsthm,mathtools} % AMS
\usepackage{icomma} % "Умная" запятая: $0,2$ --- число, $0, 2$ --- перечисление
 %% Номера формул
\mathtoolsset{showonlyrefs=true} % Показывать номера только у тех формул, на которые есть \eqref{} в тексте.

\author{Батарин Егор}
\title{Саморепродукция}
\date{\today}
 
\begin{document} % конец преамбулы, начало документа
 
\maketitle
 
\begin{abstract}
   Цель работы: изучения явления саморепродукции и применение его к измерению праметров периодических структур.
\end{abstract}

\section{Теория}

При дифрации на предмете с периодической структурой наблюдается явление саморепродукции: на некотором расстоянии от предмета вдоль волны направления распространения волны появляется изображение, которое потом периодически повторяется. Покажем, почему такой эффект имеет место быть:

Выражение для плоской монохроматической волны имеет вид:
\[ E(\vec{r}; t) = a_0e^{-i(\omega t - \vec{k}\vec{r}-\psi_0)} \] 
Здесь $a_0$ - действительное число, $\vec{k}\vec{r} = ux + vy + \sqrt{k^2-u^2-v^2}\cdot z$. Будем в дальнейшем опускать зависимость от времени $e^{-i\omega t}$. Тогда комлексная амплитуда запишется в виде:
\[ f(x,y,z) = a_0e^{i\psi_0}e^{i(ux+vy)}e^{i\sqrt{k^2-u^2-v^2}\cdot z} = f(x,y,0)e^{i\sqrt{k^2-u^2-v^2}\cdot z}\]
Пусть плоская волна падает на транспорант, описываемый функцией $t(x,y)$ (рассмотрим, для простоты, одномерный случай $t(x,y) = t(x)$, положим $y=0$). Если комплексная амплитуда на входе равна $a_0e^{i\psi_0}$, то на выходе получится $a_0e^{i\psi_0}t(x)$. 
\newpage
Считая транспорант периодической структурой, применим теорему Фурье:
\[ f(x, 0_+) = a_0e^{i\psi_0}t(x) = a_0 + \sum_{n=1}^{\infty} a_n\cos{(nu_nx)} + b_n\sin{(nu_nx)}  = \sum_{n=-\infty}^{\infty} c_ne^{iu_nx} = \sum_{n=-\infty}^{\infty} c_ne^{i\frac{2\pi}{d}x} \]
Тогда решение уравнения Гельмгольца будет иметь вид:
\[f(x,z) = \sum_{n=-\infty}^{\infty} c_ne^{iu_nx}e^{i\sqrt{k^2-u^2_n}\cdot z}\]
Каждая плоская волна в данной сумме приобрела при распространении от транспоранта до плоскости $z = \textrm{const}$ набег фазы равный 
\[\phi_n = \sqrt{k^2-u_n^2}\cdot z \approx kz- \frac{u^2_n}{2k}z\]
Положим $z = z_n = \frac{2d^2}{\lambda}\cdot N$, тогда $\frac{u^2_n}{2k}z = 2\pi\cdot p$, где $p$ - целое число, поэтому получим:
\[f(x,z) = e^{ikz}\cdot f(x,0_+)\]
Отсюда получаем, что поле волны в плоскости $z = \textrm{const}$ полностью повторяет структуру поля волны в плоскости $z = 0_+$, отличаясь лишь на фазовый множитель $e^{ikz}$.
\section{Выполнение} 
\end{document} % конец документа